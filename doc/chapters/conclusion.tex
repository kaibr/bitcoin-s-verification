\chapter{Conclusion}
\label{chap:conclusion}

Because of the limitations of the verication tool, we could only verify a rewritten version of the original Bitcoin-S code.
So we can not guarantee the correctness of the addition of Satoshis with zero in Bitcoin-S.
Not all changes we made were as trivial as the replacement of objects with case objects.
For these non-trivial changes, as seen for example the bound check in section \ref{sec:bound_check}, we cannot say whether they are equivalent to the original implementation or not.

So code should be written specically with formal verication in mind, in order to successfully verify it.
Otherwise, it needs a lot of changes in the software because verification is mathematical and the current software is written mostly in object-oriented style.
Software written in the functional paradigm would be much easier to reason about.

Thus, either Stainless must find ways to translate more of built-in object-oriented patterns of Scala to their verification tool or developers must invest more in functional programming.

Also, we found that trying to verify code reveals bugs as shown in section \ref{sec:bugfix}.
Finally, our work led to some feedback to the Stainless developers to improve the tool.
