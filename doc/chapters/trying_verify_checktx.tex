\chapter{Trying to Verify the no Inflation Property}
\label{chap:verify_check}

This chapter describes the part of Bitcoin-S needed to verify the No Inflation Property described before.
We are going to create a transaction and show you the relevant parts of the method checkTransaction, where transactions are checked against some properties.
Then we see some troubles occurred during this work.
In the end you will see how we fixed a bug in Bitcoin-S.


\section{Creation of a Transaction}

Some of the code in this section is copied or adapted from Bitcoin-S-Core transaction builder example.\cite{BitcoinSCore:txbuilderexample}
Bitcoin-S-Core has a bitcoin transaction builder class with the following constructor:
\begin{lstlisting}[language=scala]
  BitcoinTxBuilder(
    destinations: Seq[TransactionOutput], // where to send the money
    utxos: BitcoinTxBuilder.UTXOMap,      // unspent transaction outputs
    feeRate: FeeUnit,                     // fee rate per byte
    changeSPK: ScriptPubKey,              // where to send the change
    network: BitcoinNetwork               // bitcoin network information
  ): Future[BitcoinTxBuilder]
\end{lstlisting}

To return a Future does not make sense, since the implementing constructor calls either Future.successfull or Future.fromTry which returns an already resolved Future.
This might be for future purposes.

Now we create a transaction.

First, we need a some money.
Thus, we create a fake transaction with one single output.
This transaction can be parsed from the bitcoin network but we create one manually in order to see this process.
\begin{lstlisting}[language=scala]
  val privKey = ECPrivateKey.freshPrivateKey
  val creditingSPK = P2PKHScriptPubKey(pubKey = privKey.publicKey)

  val amount = Satoshis(Int64(10000))

  val utxo = TransactionOutput(currencyUnit = amount, scriptPubKey = creditingSPK)

  val prevTx = BaseTransaction(
    version = Int32.one,
    inputs = List.empty,
    outputs = List(utxo),
    lockTime = UInt32.zero
  )
\end{lstlisting}

On line one and two we create a new keypair to sign the next transaction and have a scriptPubKey where the bitcoins are.
This is our keypair.
So the money is transferred to our public key.
Line four specifies the amount of satoshis we have in the transaction.
Then we create the actual transaction from line 6 to 13.

Now that we have some bitcoins, we create the new transaction where we want to spend them.

First, we need some out points.
They point to outputs of previous transactions.
We use the index zero, because the previous transaction has only one output that becomes the first index zero.
If there were two previous outputs, the second output would become the index 1 and so on.
\begin{lstlisting}[language=scala]
  val outPoint = TransactionOutPoint(prevTx.txId, UInt32.zero)

  val utxoSpendingInfo = BitcoinUTXOSpendingInfo(
    outPoint = outPoint,
    output = utxo,
    signers = List(privKey),
    redeemScriptOpt = None,
    scriptWitnessOpt = None,
    hashType = HashType.sigHashAll
  )

  val utxos = List(utxoSpendingInfo)
\end{lstlisting}

This utxs are the inputs of our transaction.

Second, we need destinations to spend the bitcoins to.
For the sake of convenience we create only one.
\begin{lstlisting}[language=scala]
  val destinationAmount = Satoshis(Int64(5000))

  val destinationSPK = P2PKHScriptPubKey(pubKey = ECPrivateKey.freshPrivateKey.publicKey)

  val destinations = List(
    TransactionOutput(currencyUnit = destinationAmount, scriptPubKey = destinationSPK)
  )
\end{lstlisting}

We spend 5000 satoshis to the newly created random public key.

Finally, we define the fee rate in satoshis per one byte transaction size as well as some bitcoin network parameters.
The bitcoin network parameters are not important so we use some static values for testing purpose.
\begin{lstlisting}[language=scala]
  val feeRate = SatoshisPerByte(Satoshis.one)

  val networkParams = RegTest // some static values for testing
\end{lstlisting}

Now lets build the transaction with those data.
\begin{lstlisting}[language=scala]
  val txBuilderF: Future[BitcoinTxBuilder] = BitcoinTxBuilder(
    destinations = destinations, // where to send the money
    utxos = utxos,               // unspent transaction outputs
    feeRate = feeRate,           // fee rate per byte
    changeSPK = creditingSPK,    // where to send the change
    network = networkParams      // bitcoin network information
  )

  val signedTxF: Future[Transaction] = txBuilderF
    .flatMap(_.sign)                       // call sign on the transaction builder
    .map {
      (tx: Transaction) => println(tx.hex) // transaction in hex for the bitcoin network
    }
\end{lstlisting}

Line one to seven creates a transaction builder which is then signed on line ten.
We can now use our transaction object on line twelve.
For example, after calling \emph{hex} on it, we can send the returned string to the bitcoin network.


\section{Validation of a Transaction}

Bitcoin-S offers a function called \emph{checkTransaction} located in the ScriptInterpreter object.
This is its type signature:
\begin{lstlisting}[language=scala]
  checkTransaction(transaction: Transaction): Boolean
\end{lstlisting}

We can pass a transaction and it returns a Boolean whether the transaction is valid or not.
So for example when we pass the transaction built before the returned value would be true, because it's a valid transaction.
It might not be accepted by the bitcoin network but for a transaction on its own its valid.
We can not check context with it, because we can only pass one transaction.

There are several checks in checkTransaction.
For example, it checks if there is either no input or no output.
In this case we get false.

The relevant part for the bug we found:
\begin{lstlisting}[language=scala]
  val prevOutputTxIds = transaction.inputs.map(_.previousOutput.txId)
  val noDuplicateInputs = prevOutputTxIds.distinct.size == prevOutputTxIds.size
\end{lstlisting}

It gathers all transaction ids referenced by the out points.
When we call \emph{distinct} on the returned list, we get a list with duplicate removed.
If the size of the new list is the same as the size of the old, we know that there was no duplicate transaction id, because, as said, distinct removes the duplicates.


\section{Fixing a Bug in Bitcoin-S}

We can see that there is a bug in the checkTransaction function from before, recognized and fixed through this work.

\begin{figure}[H]
	\centering
		\includegraphics[scale=0.5]{images/bitcoin-s-pr-comment.png}
	\caption{Nice catch comment on our PR \#435 on GitHub}
	\label{fig:output1}
\end{figure}

Here the relevant code of checkTransaction again:
\begin{lstlisting}[language=scala]
  val prevOutputTxIds = transaction.inputs.map(_.previousOutput.txId)
  val noDuplicateInputs = prevOutputTxIds.distinct.size == prevOutputTxIds.size
\end{lstlisting}

What happens, if we have two TransactionOutPoints (previousOutputs) with a different index but referencing the same Transaction ID (txId)?

According to the Bitcoin protocol this is possible.
A transaction can have multiple outputs that should be referenceable by the next transaction.
So this is clearly a bug.

What should not be possible is a transaction referencing the same output twice.
This bug occured in Bitcoin Core known as CVE-2018–17144 which was patched on September 18, 2018. \cite{cve201817144}

Here, Bitcoin-S did a bit too much and marked all transaction as invalid, if they referenced the same transaction twice.
The fix is, to check on TransactionOutPoint instead of TransactionOutPoint.txId, because TransactionOutPoint contains the txId as well as the output index it references.
So in pseudo code, we check on the tuple (tx, index) instead of (tx).
The fixed code:
\begin{lstlisting}[language=scala]
  val prevOutputs = transaction.inputs.map(_.previousOutput)
  val noDuplicateInputs = prevOutputs.distinct.size == prevOutputs.size
\end{lstlisting}

Since TransactionOutPoint is a case class and Scala has a built in == for case classes there is no need to implement TransactionOutPoint.==.
\begin{figure}[H]
	\centering
		\includegraphics[scale=0.396]{images/bitcoin-s-pr.png}
	\caption{Line changes for PR \#435 from GitHub}
	\label{fig:output1}
\end{figure}

This was fixed in \href{https://github.com/bitcoin-s/bitcoin-s/pull/435}{pull request \mypound435} on GitHub at April 23, 2019 through this work along a unit test to prevent this bug from appearing again in the future.


\section{Adjusting Property}

Trying to integrate Stainless in Bitcoin-S caused a lot of troubles, because of version conflicts.
If you are interested in it you can read more in the appendix \ref{chap:appendix_arb}.

It takes too much time to do it and after some discussion, we decided to extract only the parts of the code that are needed for checkTransaction.

The extracted code has more than 1500 lines.
After running Stainless on it, it throws a really huge bunch of errors about what Stainless can not reason about.
And after fixing some of those errors there appear new ones as we can see in the next chapter.
This would require to change nearly everything of the extracted code.

We arrived at the decision that we need a smaller part to verify.
We have chosen to verify the addition with zero of Bitcoin-Ss CurrencyUnit class as described in the next chapter.
