\chapter*{Abstract}
\label{chap:abstract}


In this thesis we experiment in formal verification.
To this end, we verify for correctnes a part of the implementation of the Bitcoin protocol in Bitcoin-S using the verification framework Stainless.
During the work we have faced some challenges.

First of all, we indentify a property of the Bitcoin protocol which we want to verify.
Because of the found bug in the C++ implementation of Bitcoin Core, which allowed a non-coinbase transaction to generate new coins, this property has been chosen.

The first challenging step is to integrate Stainless into Bitcoin-S.
The difference in the versions of the sbt framework and Scala used in Bitcoin-S and Stainless makes the integration process quite diffucult.
Thus, we have extracted the classes implementing the property mentioned above from the subpackage Bitcoin-S-Core.

To understand the building process of a transaction, we experiment with the transaction creation and validation implemented in Bitcoin-S-Core.
Furthermore, we analyse function \textit{checkTransaction} which is important for the protection against the introduced missuse of the Bitcoin protocol.

The next challenging stage is to use Stainless for verification of this function.
Because Stainless takes as an input the functional subset of Scala with some imperative features we have to reimplement the extracted code according to the rules of Stainless.
All extracted classes consist of more than 1500 lines of code what takes too much time to rewrite it. 
Not to mention, this is only a preparation phase towards verification and not verification itself.
So, we have decided to verify another functionality implemented in the Bitcoin-S-Core package which requires the extraction of the smaller batch of classes.
The coin addition seems to be a good candidate.
The operation of the coin addition with zero should result to the same value of the coin.
We rewrite the code of 2 classes implementing this functionality accoring to the functional style.
Then we define a formal specification for the function of the addition and some other dependent functions.

During this work, we have detected a bug in the function \textit{checkTransaction} in Bitcoin-S-Core which forbids one type of a valid transaction.
We have fixed it, and our pull request has been merged to the Bitcoin-S project.
Furthermore, we have verified the coin addition with zero using Stainless.
The framework has reported the operation as valid.
But we had to transform the original code too much. So that we cannot state that the code of Bitcoin-S has been successfully verified.

