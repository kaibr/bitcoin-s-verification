
\documentclass[hyphens, a4paper,USenglish,cleveref, autoref, thm-restate]{oasics-v2019}
%for section-numbered lemmas etc., use "numberwithinsect"
%for enabling cleveref support, use "cleveref"
%for enabling autoref support, use "autoref"
%for anonymousing the authors (e.g. for double-blind review), add "anonymous"
%for enabling thm-restate support, use "thm-restate"

\graphicspath{{./images/}}

\bibliographystyle{plainurl}
% the mandatory bibstyle

\title{Towards Verifying the Bitcoin-S Library} 

% \titlerunning{Dummy short title}
% optional, please use if title is longer than one line

\author{Ramon Boss}{Bern University of Applied Sciences, Switzerland}{ramon.boss@outlook.com}{}{}
\author{Kai Brünnler}{Bern University of Applied Sciences, Switzerland}{kai.bruennler@bfh.ch}{}{}
\author{Anna Doukmak}{Bern University of Applied Sciences, Switzerland}{anna.doukmak@gmail.com}{}{}

\authorrunning{R. Boss and K. Brünnler and A. Doukmak}
% mandatory. First: Use abbreviated first/middle names. Second (only in severe cases): Use first author plus 'et al.'

\Copyright{Ramon Boss and Kai Brünnler and Anna Doukmak}
% mandatory, please use full first names. OASIcs license is "CC-BY";  http://creativecommons.org/licenses/by/3.0/

\begin{CCSXML}
<ccs2012>
<concept>
<concept_id>10003752.10003790.10002990</concept_id>
<concept_desc>Theory of computation~Logic and verification</concept_desc>
<concept_significance>500</concept_significance>
</concept>
</ccs2012>
\end{CCSXML}

\ccsdesc[500]{Theory of computation~Logic and verification}


\keywords{Bitcoin, Scala, Bitcoin-S, Stainless}
% mandatory; please add comma-separated list of keywords

\category{Short Paper}

% \relatedversion{A full version of the paper is available at
%   \url{https://github.com/kaibr/bitcoin-s-verification/blob/master/paper/paper.pdf}.}

% \supplement{}
% optional, e.g. related research data, source code, ... hosted on a repository like zenodo, figshare, GitHub, ...

% \funding{(Optional) general funding statement
% \dots}%optional, to capture a funding statement, which applies to all authors.
%Please enter author specific funding statements as fifth argument of the \author macro.

%\acknowledgements{I want to thank \dots}%optional

\nolinenumbers %uncomment to disable line numbering

% \hideOASIcs %uncomment to remove references to OASIcs series (logo, DOI, ...),
% e.g. when preparing a pre-final version to be uploaded to arXiv or another public repository


\EventEditors{Bruno Bernardo and Diego Marmsoler}
\EventNoEds{2}
\EventLongTitle{2nd Workshop on Formal Methods for Blockchains (FMBC 2020)}
\EventShortTitle{FMBC 2020}
\EventAcronym{FMBC}
\EventYear{2020}
\EventDate{July 20--21, 2020}
\EventLocation{Los Angeles, California, USA (Virtual Conference)}
\EventLogo{}
\SeriesVolume{84}
\ArticleNo{7}


\usepackage{listings, color}


\definecolor{dkgreen}{rgb}{0,0.6,0}
\definecolor{gray}{rgb}{0.5,0.5,0.5}
\definecolor{mauve}{rgb}{0.58,0,0.82}


\lstdefinestyle{scala}{
  language=scala,
  basicstyle=\footnotesize\ttfamily,
%  basicstyle=\scriptsize\ttfamily, % kai: temporary 
  breaklines=true,
  keywordstyle=\color{blue},
  commentstyle=\color{dkgreen},
  numbers=left,
  %frame=single, % Border around box
  %numbersep=-7pt,
  numberstyle=\color{gray},
  stringstyle=\color{mauve}
}

\lstdefinestyle{stainless}{
  numbers=none,
  breaklines=true,
  breakautoindent=true,
  breakindent=63pt,
  basicstyle=\footnotesize\ttfamily
}

% make paragraph headings bold instead of italics
\renewcommand{\paragraph}{\textbf}%

% The default bold(bx) is too wide
% Choose a nicer narrower bold
\DeclareFontSeriesDefault[rm]{bf}{b}


\begin{document}

\maketitle

\begin{abstract}
  We try to verify properties of the Bitcoin-S library, a Scala
  implementation of parts of the Bitcoin protocol. We use the
  Stainless verifier which supports programs in a fragment of Scala
  called \emph{Pure Scala}. Since Bitcoin-S is not written in this
  fragment, we extract the relevant code from it and rewrite it until
  we arrive at code that we successfully verify. In that process we
  find and fix two bugs in Bitcoin-S.
\end{abstract}

\section{Introduction}

For software handling cryptocurrency, correctness is clearly crucial.
However, even in very well-tested software such as Bitcoin Core,
serious bugs occur. The most recent example is the bug found in
September 2018 \cite{cve201817144} which essentially allowed to
arbitrarily create new coins. Such software is thus a worthwhile
target for formal verification. In this work, we set out to verify
properties of the Bitcoin-S library with the Stainless verifier. So
this is a case study in applying the Stainless verifier to existing
real-world code.

\paragraph{The Bitcoin-S Library.} The Bitcoin-S library is an
implementation of parts of the Bitcoin protocol in Scala
\cite{BitcoinS:website,BitcoinS:github}. In particular, it allows to
serialize, deserialize, sign and validate Bitcoin transactions. The
library uses immutable data structures and algebraic data types but is
not specifically written with formal verification in mind. According
to the website, the library is used in production, handling
significant amounts of cryptocurrency each day
\cite{BitcoinS:website}.

\paragraph{The Stainless Verifier.} Stainless is the successor of the
Leon verifier and is developed at EPF Lausanne
\cite{DBLP:conf/ecoop/BlancKKS13,DBLP:conf/pldi/VoirolKK15,DBLP:conf/pldi/BlancK15}.
A distinguishing feature of Stainless is that it accepts
specifications written in the programming language itself
(Scala). Also, it focusses on counterexample finding in addition to
proving correctness. Counterexamples are immediately useful to
programmers, which can not be said about correctness proofs.

The example in Figure~\ref{fig:factorial} is adapted from the Stainless
documentation \cite{Stainless:documentation} and shows how the verifier is
used. Note how a precondition is specified using \texttt{require} and
a postcondition using \texttt{ensuring}. Our function does not satisfy
the specification. An overflow in the 32-bit integer type leads to a
negative result for the input 17, as Stainless reports in
Figure~\ref{fig:failed}. Changing the type \texttt{Int} to
\texttt{BigInt} will result in a successful verification.


\begin{figure}
\begin{lstlisting}[style=scala]
  def factorial(n: Int): Int = {
      require(n >= 0)
      if (n == 0) {
        1
      } else {
        n * factorial(n - 1)
      }
  } ensuring(res => res >= 0)
\end{lstlisting}
	\caption{Factorial function with specification}
	\label{fig:factorial}
\end{figure}

\begin{figure}
	\centering
		\includegraphics[width=\textwidth]{output1.png}
	\caption{Stainless output for the factorial function}
	\label{fig:failed}
\end{figure}


\paragraph{The Pure Scala Fragment.} The Scala fragment supported by
Stainless comprises algebraic data types in the form of abstract
classes, case classes and case objects, objects for grouping classes
and functions, boolean expressions with short-circuit interpretation,
generics with invariant type parameters, pattern matching, local and
anonymous classes and more.  In addition to Pure Scala Stainless also
supports some imperative features, such as while loops and using a
(mutable) variable in a local scope of a function. They turn out not
to be relevant for our current work.

What will turn out to be more relevant for us are the Scala features
which Stainless does not support, such as: inheritance by objects,
abstract type members, and inner classes in case objects. Also,
Stainless has its own library of some core data types and functions
which are mapped to corresponding data types and functions inside of
the SMT solver that Stainless ultimately relies on. Those data types
in general do not have all the methods of the Scala data types. For
example, the \texttt{BigInt} type in Scala has methods for bitwise
operations while the \texttt{BigInt} type in Stainless does not.

\paragraph{Outline and Properties to Verify.} In the next section we
try to verify the property that a regular (non-coinbase) transaction
can not generate new coins. We call it the \emph{No-Inflation
  Property}. Trying to verify it, we uncover and fix a bug in the
Bitcoin-S library. We then find that there is too much code involved
that lies outside of the supported fragment to currently make this
verification feasible. So we turn to a simpler property to verify. The
simplest possible property we can think of is the fact that adding
zero satoshis to a given amount of satoshis yields the given amount of
satoshis. We call it the \emph{Addition-With-Zero Property} and we try
to verify it in Section~3. Here as well we see that a significant part
of the code lies outside of the supported fragment. We rewrite it
until we arrive at code that we successfully verify. In that process
we find and fix a second bug in Bitcoin-S.


\section{The No-Inflation Property}

\paragraph{An Attempt at Verification.} Naively trying Stainless on
the entire Bitcoin-S codebase results in many errors -- as was to be
expected. We tried to extract only the code relevant to the
No-Inflation Property and to verify that. However, even the extracted
code has more than 1500 lines and liberally uses Scala features
outside of the supported fragment. We started to rewrite the code in
the supported fragment, but quickly realized that a better approach is
to first verify a simpler property depending on less code and later
come back to the No-Inflation Property with more experience. However,
during the process of trying to rewrite the code, we found a bug in
the \texttt{checkTransaction} function shown in
Figure~\ref{fig:checktrans}.

\paragraph{A Bug in the checkTransaction Function.} Given a
transaction the function returns true if some basic checks succeed,
otherwise false. For example, one of those checks is that both the
list of inputs and list of outputs need to be non-empty.


\begin{figure}
\begin{lstlisting}[style=scala]
def checkTransaction(transaction: Transaction): Boolean = {
  val inputOutputsNotZero =
    !(transaction.inputs.isEmpty || transaction.outputs.isEmpty)
  val txNotLargerThanBlock = 
    transaction.bytes.size < Consensus.maxBlockSize
  val outputsSpendValidAmountsOfMoney = 
    !transaction.outputs.exists(o =>
      o.value < CurrencyUnits.zero || o.value > Consensus.maxMoney)

  val outputValues = transaction.outputs.map(_.value)
  val totalSpentByOutputs: CurrencyUnit =
    outputValues.fold(CurrencyUnits.zero)(_ + _)
  val allOutputsValidMoneyRange = 
    validMoneyRange(totalSpentByOutputs)
  val prevOutputTxIds = transaction.inputs.map(_.previousOutput.txId)
  val noDuplicateInputs = 
    prevOutputTxIds.distinct.size == prevOutputTxIds.size

  val isValidScriptSigForCoinbaseTx = transaction.isCoinbase match {
    case true =>
      transaction.inputs.head.scriptSignature.asmBytes.size >= 2 &&
        transaction.inputs.head.scriptSignature.asmBytes.size <= 100
    case false =>
      !transaction.inputs.exists(
        _.previousOutput == EmptyTransactionOutPoint)
  }
  inputOutputsNotZero && txNotLargerThanBlock && 
  outputsSpendValidAmountsOfMoney && noDuplicateInputs &&
  allOutputsValidMoneyRange && noDuplicateInputs && 
  isValidScriptSigForCoinbaseTx
}
\end{lstlisting}
  
  \caption{The \texttt{checkTransaction} function}
  \label{fig:checktrans}
\end{figure}

\begin{figure}
\begin{lstlisting}[style=scala, firstnumber=15]
  val prevOutputs = transaction.inputs.map(_.previousOutput)
  val noDuplicateInputs = 
    prevOutputs.distinct.size == prevOutputs.size
\end{lstlisting}
  \caption{Bug Fix}
  \label{fig:bugfix1}
\end{figure}


Note particularly lines 15-17.  Here, the value
\texttt{prevOutputTxIds} gathers a list of all transaction identifiers
referenced by the inputs of the current transaction. If the size of
this list is the same as the size of this list with duplicates
removed, we know that no transaction has been referenced twice. This
prevents a transaction from spending two different outputs of the same
previous transaction. The check is too strict:
\texttt{checkTransaction} returns false for valid transactions.

The fix is simple: we perform the duplicate check on the
\texttt{TransactionOutPoint} instances instead of on their transaction
identifiers. Note that \texttt{TransactionOutPoint} is a case class
and thus its notion of equality is just what we need: equality of of
both the transaction identifier and the output index.

Specifically, we replace lines 15-17 as shown in
Figure~\ref{fig:bugfix1}. We submitted this fix together with a
corresponding unit test to the Bitcoin-S project in a pull request,
which has been merged \cite{BitcoinS:pull435}.

We now turn to the much simpler Addition-With-Zero Property.





\section{The Addition-With-Zero Property}

It is of course a crucial property we are verifying here: if zero
satoshis were credited to your account, you would not want your
balance to change! It is also the simplest meaningful property to
verify that we can think of. However, the code involved in performing
the addition of two satoshi amounts in Bitcoin-S is non-trivial. The
reason for that is a peculiarity of consensus code: agreement with the
majority is the only relevant notion of correctness. The most widely
used bitcoin implementation by far is the reference implementation
Bitcoin Core, written in C++. For consensus code, Bitcoin-S thus has
little choice but to be in strict agreement with the reference
implementation. To achieve that, it implements C-like data types in
Scala and then implements functionality using those C-like data
types. For example, the Satoshis class, which represents an amount of
satoshis, is implemented using the class \texttt{Int64} which aims to
represent the C-type \texttt{int64\_t}.

\begin{figure}
\lstinputlisting[style=scala]{../code/addition/src/main/scala/extracted/number/NumberType.scala}
  \caption{Extracted Code from NumberType.scala}
  \label{fig:numbertype}
\end{figure}

\begin{figure}
\lstinputlisting[style=scala]{../code/addition/src/main/scala/extracted/currency/CurrencyUnits.scala}
  \caption{Extracted Code from CurrencyUnits.scala}
  \label{fig:currencyunits}
\end{figure}

\paragraph{Extracting the Relevant Code.} The relevant code for the
addition of satoshis is in two files: CurrencyUnits.scala and
NumberType.scala. From those files we removed the majority of the code
because it is not needed for the verification of our property. For
example, we removed all number types except for \texttt{Int64} (so
\texttt{Int32}, \texttt{UInt64}, etc.) because they are not used. We
also removed the superclasses \texttt{Factory} and
\texttt{NetworkElement} of \texttt{CurrencyUnit} and \texttt{Number},
respectively, because the inherited members are not used. We further
removed all binary operations on \texttt{Number} that are not used,
like subtraction and multiplication. The extracted code is shown in
Figure~\ref{fig:numbertype} and Figure~\ref{fig:currencyunits}.

\paragraph{A Bug in the checkResult Function.} Note the
\texttt{checkResult} function on line 12 and the value
\texttt{andMask} on line 23 of NumberType.scala. The function is
intended to catch overflows by performing a bitwise conjunction of its
argument with \texttt{andMask} and comparing the result with the
argument. However, because of the way Java BigIntegers are represented
\cite{wikipedia:twocomp} and because bitwise operations implicitly
perform a sign extension \cite{java:bigint} on the shorter operand,
the function does not actually catch overflows.

While this is a potentially serious bug, it turns out that
\texttt{checkResult} is only ever called inside a constructor call for
a number type which contains the intended range check, see lines
32-35. The \texttt{checkResult} function thus can, and should, be
removed entirely. The Bitcoin-S developers have acknowledged the bug
and we submitted a pull request to fix it, which has been merged
\cite{BitcoinS:pull565}.

For further development of Bitcoin-S, this raises a question. If the
goal of the \texttt{Int64} type is to emulate \texttt{int64\_t} then
why does it prevent overflows? To achieve strict agreement with
Bitcoin Core, a better approach might be to remove overflow checking
from the data type and to add it in exactly those places where it
happens in Bitcoin Core.

\paragraph{Rewriting the Code.} We now turn to the list of Scala
features used by the extracted code which are not supported by
Stainless and how to rewrite the code in the supported fragment.

All code changes are \emph{equivalent} in the (admittedly narrow)
sense that if the Addition-With-Zero Property holds for the rewritten
code, then it also holds for the original code.

\paragraph{Inheriting Objects.} In both files we have objects
extending the BaseNumbers trait, on lines 30 and 23 respectively,
which Stainless does not support. We simply turn those objects into
case objects. That code is equivalent: case objects have various
additional properties (for example, being serializable) but none of
our code depends on the absence of those.

\paragraph{Abstract Type Members.} In CurrencyUnits.scala on line 6
there is an abstract type that is not supported. Note that we can not
simply replace it with a (supported) type parameter since the
CurrencyUnit class uses one of its implementing classes:
Satoshis. Since the Satoshis class overrides \texttt{A} with
\texttt{Int64} anyway, we just remove the abstract type declaration
and replace \texttt{A} by \texttt{Int64} everywhere.

\paragraph{Non-Literal BigInt Constructor Argument.} In
CurrencyUnits.scala on line 18 the BigInt constructor is called with a
non-literal argument. As described before, the types in the Stainless
library are more restricted than their Scala library counterparts. In
particular, the Stainless BigInt constructor is restricted to literal
arguments. So we simply replace \texttt{toLong} by
\texttt{underlying.toBigInt}: instead of converting the underlying
\texttt{Int64} (which in turn has an underlying \texttt{BigInt}) to
\texttt{Long} and then back to \texttt{BigInt} we simply directly
return the \texttt{BigInt}. This is an equivalent transformation: the
only thing that might go wrong in the detour via \texttt{Long} is that
the underlying \texttt{BigInt} does not fit into a
\texttt{Long}. However, the only constructor of \texttt{Int64Impl}
ensures exactly that and all functions producing \texttt{Int64} do so
via this constructor.

\paragraph{Self-Reference in Type Parameter Bound.} In
NumberTypes.scala both on lines 3 and 19 is a class with a type
parameter and a type boundary that contains that type parameter
itself. Stainless does not currently support such self-referential
type boundaries. We opened an issue \cite{Stainless:issue519} on the
Stainless repository and the developers have targeted version 0.4 to
support self-referential type boundaries. Since our code only uses
Number with type parameter \texttt{T} instantiated to \texttt{Int64},
we just remove the type parameter declaration and replace all its
occurrences by \texttt{Int64}.

\paragraph{Missing Member \texttt{bigInteger} in BigInt.} In
NumberType on line 6 there is a reference to \texttt{bigInteger}. The
Scala \texttt{BigInt} class is essentially a wrapper around
\texttt{java.math.BigInteger}. \texttt{BigInt} has a member
\texttt{bigInteger} which is the underlying instance of the Java
class. The Java class has a method \texttt{longValueExact} which
returns a \texttt{long} only if the \texttt{BigInteger} fits into a
\texttt{long}, otherwise throws exception. Stainless does not support
Java classes and in particular its \texttt{BigInt} has no member
\texttt{bigInteger}. However, our code does not call \texttt{toLong}
anymore, so we just remove it.

\paragraph{Type Members.} In NumberType.scala there is a type member
on line 4. Our version of Stainless (0.1) does not support type
members. We just remove the declaration and replace all occurrences of
\texttt{A} with \texttt{BigInt}, since \texttt{A} is never overwritten
in an implementing class.  Note that in the meantime Stainless has
implemented support for type members \cite{Stainless:pull470}. Since
version 0.2 verification should succeed without this change.


\paragraph{Missing Bitwise-And Method on BigInt.} Contrary to Scala
\texttt{BigInt}, the Stainless \texttt{BigInt} class does not support
bitwise operations, in particular not the \&-method used in
NumberType.scala on line 13. However, as described above, the
\texttt{checkResult} function is both broken and redundant, so we
remove it and all calls to it.

\paragraph{Inner Class in Case Object.} We have inner classes in
NumberType.scala on line 31 and in CurrencyUnits.scala on line
26. Stainless does not support inner classes in a case object. We just
move the inner classes out of the case objects. They do not interfere
with any other code.

\paragraph{Message Parameter in Require.} The calls of the require
function on lines 32 and 34 in CurrencyUnits.scala have a second
parameter: the error message. Stainless does not support the message
parameter. We simply remove it.

\paragraph{Missing Implicit Long to BigInt Conversion.} The Scala
\texttt{BigInt} class has implict conversions from \texttt{Long} which
NumberType.scala uses on lines 32 and 34. They are missing in the
Stainless \texttt{BigInt}. A \texttt{BigInt} constructor with a
\texttt{Long} argument is also missing. We thus replace the
\texttt{Long} literals by an explicit call to the \texttt{BigInt}
constructor with a literal string argument,
e.g. \texttt{BigInt("-9223...5808")}.


\paragraph{The Specification.} Now that all our code is in the
supported fragment, we can finally write our specification. We add a
postcondition to the +-method of the \texttt{CurrencyUnit}-class
(Figure~\ref{fig:currencyunits}, lines 9-11) resulting in
Figure~\ref{fig:spec}. We successfully verify it with Stainless, as
the output in Figure~\ref{fig:result} shows.

The original Bitcoin-S code we started from, the extracted code, and
the finally verified code are available in our GitHub repository \cite{verification:github}.

\begin{figure}
  \lstinputlisting[style=scala, firstline=9, lastline=12,
  firstnumber=9]{../code/addition/src/main/scala/verified/currency/CurrencyUnits.scala}
\caption{Addition function with specification}
\label{fig:spec}
\end{figure}

\begin{figure}
	\centering
		\includegraphics[width=\textwidth]{result_output}
	\caption{Stainless output for the rewritten code}
  \label{fig:result}
\end{figure}


\section{Conclusion and Future Work}

We are happy to see some friendly green verifier output. However,
apart from the bugs we found, the main conclusion of this work is that
we had to non-trivially transform even a very small portion of the
code (70 lines) in order to verify it. And that was true even though
the code was purely functional to begin with. At the moment, it is
probably unrealistic to routinely formally verify properties as part
of the Bitcoin-S development process. However, Stainless development
has already progressed (e.g.\ type members are supported in recent
versions) and continues to do so (e.g. self-referential type bounds
are on the roadmap). Some missing features that we identified are
presumably very easy to support, like the message parameter in the
require function. Some other features presumably require more
substantial work, like bitwise operations on integer types.

On the other hand, Bitcoin-S uses features that might not be supported
even by future Stainless versions, such as calls to Java code.

Given our experience, the best route towards integrating verification
into the Bitcoin-S development process would be to re-implement parts
of the library in Pure Scala. We would split the library into a
\emph{verified} and \emph{non-verified} part, and use Stainless only
on the \emph{verified} part. It is then both a technical but also a
political question how much code, if any, can be moved to the
\emph{verified} part. That is an interesting direction for future
work.

\appendix

\bibliography{bibliography}


\end{document}
